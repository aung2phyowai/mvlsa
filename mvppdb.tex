%
% File acl2014.tex
%
% Contact: koller@ling.uni-potsdam.de, yusuke@nii.ac.jp

\documentclass[11pt]{article}
\usepackage{acl2014}
\usepackage{times}
\usepackage{url}
\usepackage{latexsym}

%\setlength\titlebox{5cm}

% You can expand the titlebox if you need extra space
% to show all the authors. Please do not make the titlebox
% smaller than 5cm (the original size); we will check this
% in the camera-ready version and ask you to change it back.


\title{Multi View approach for Paraphrasing OR Supervised embeddings
  for improved performance}

%% \author{Pushpendre Rastogi \\
%%   Johns Hopkins University \\
%%   {\tt pushpendre@jhu.edu} \\\And
%%   Second Author \\
%%   Affiliation / Address line 1 \\
%%   Affiliation / Address line 2 \\
%%   Affiliation / Address line 3 \\
%%   {\tt email@domain} \\}

\date{}

\begin{document}
\maketitle
\begin{abstract}
  Sentence aligned bilingual text corpora and Unannotated Monolingual
  text corpora can be used to learn large collections of semantically
  similar paraphrases. These two data sources can be modelled as two
  views over underlying latent representation of the words and
  phrases.

  OR
  
  We use CCA to adapt the features present in an existing collection
  of paraphrases to learn a more useful representation of the existing features.
\end{abstract}

\section{Introduction}
Bilingual text and monolingual text has been used to learn large
collections of paraphrases. \cite{bannard2005paraphrasing} introduced the trick and
then \cite{ganitkevitch2013ppdb} built upon it. The reason why these two views work is
that they complement each other. Monolingual data can't distinguish
between antonyms but bilingual data can. And bilingual data confounds
words that occur in the same sentence but monolingual data can
distinguish them based on their context. So these are two different
views over the same underlying representation which we can learn by
using CCA or other multi view learning techniques.

The main point of the paper is to learn embeddings or features
that best correlate with both the views. 

\textbf{But to check our own understanding} \textit{(I am speaking
  my mind here)} of CCA we first just use the
features that are already present in PPDB. and then learn a regression
function with and without sparsity constraints and with and without
CCA type dimensionality reduction and learn regression
coefficients. (The reason why it is obvious that there would be an
improvement is that we are specifically learning the representation to
optimize some objective, with the extra tuning knob of how many
dimensions to keep we must be able to do better than basic regression,
EVEN THOUGH BOTH ARE LINEAR!!)

\section{Previous Work}
Multi view learning and CCA have been applied previously in the following
papers.

\section{Data Used}


\section{Experiments}
The first experiment was the following. \\
Q. Can we learn reduced dimensionality embeddings which correlate with
Human semantic judgement better than the original basis ? \\
A. This is not really an experiment, more like a test. We expect an
improvement in correlation, in fact we would be very 
surprised if we can't improve and we would think there is a bug, that
is why this is a test. The reason why we are so confident is that we are actually
optimizing the representation and reducing dimensionality to learn 

To perform this experiment we do the following.\\
1. Gather likert-scale type human judgments about word similarity.\\
2. Gather PPDB features for the word-pairs and make a big matrix \\
3. Split that into train-dev-test\\
4. Train-Test plain regression coefficients of normalized data. \\
   (Basically find regression coefficients that best fit the training
data in the original vector space, with the original basis.)\\
5. Train-Test regression coefficients of CCA-type-latent-representations (We will not have to do normalization since CCA should take care of that.)

The second experiment is the following. \\
Q. Can we learn reduced dimensionality embeddings which correlate with
Monolingual alignment ?  \\




\section*{Acknowledgments}
This material is based on research sponsored DARPA under agreement
number FA8750-13-2-0017 (the DEFT program).

% include your own bib file like this:
\bibliographystyle{acl}
\bibliography{references}
\end{document}
