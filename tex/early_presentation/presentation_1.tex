%%%%%%%%%%%%%%%%%%%%%%%%%%%%%%%%%%%%%%%%%
% Beamer Presentation
% LaTeX Template
% Version 1.0 (10/11/12)
%
% This template has been downloaded from:
% http://www.LaTeXTemplates.com
%
% License:
% CC BY-NC-SA 3.0 (http://creativecommons.org/licenses/by-nc-sa/3.0/)
%
%%%%%%%%%%%%%%%%%%%%%%%%%%%%%%%%%%%%%%%%%

\documentclass{beamer}

\mode<presentation> {


\usetheme{default}
\usecolortheme{seahorse}

%\setbeamertemplate{footline} % To remove footer uncomment this line
\setbeamertemplate{footline}[page number] % To replace footer with slide count uncomment this line
\setbeamertemplate{navigation symbols}{} % To remove the navigation symbols uncomment this line
}

\usepackage{graphicx} % Allows including images
\usepackage{booktabs} % Allows the use of \toprule, \midrule and \bottomrule in tables

\title[Multi View {PPDB}]{Multi View Learning and Paraphrasing} 

\author{Pushpendre Rastogi} 
\institute[JHU] 
{
Johns Hopkins University \\ 
\medskip
\textit{pushpendre@jhu.edu} 
}
\date{} 

\begin{document}

\begin{frame}
\titlepage 
\end{frame}

\begin{frame}
\frametitle{Overview} 
\tableofcontents 
\end{frame}


%------------------------------------------------
\section{What are the Multiple Views ?}
  
\begin{frame}
\frametitle{Bilingual Data}
\begin{table}[htbp]
  \centering
  \begin{tabular}{c c c | c c c}
    W& X& Y& $\omega$& $\xi$& $\gamma$\\
    A& B& C& $\alpha$& $\beta$& $\gamma$\\
  \end{tabular}
  \caption{English Greek Sentence aligned corpus with two sentences. Y
  and C are synonyms since they translate to the same word $\gamma$}
  \label{tab:example1-1}
\end{table}
\begin{table}[htbp]
  \centering
  \begin{tabular}{c | c c c c c}
    & \textbf{$\alpha$} & \textbf{$\beta$} & \textbf{$\gamma$} &
    \textbf{$\omega$} & \textbf{$\xi$} \\
    \hline
    A& 2& 1& 1& 0& 0\\ 
    B& 1& 2& 1& 0& 0\\
    C& 1& 1& 2& 0& 0\\
    W& 0& 0& 1& 2& 1\\
    X& 0& 0& 1& 1& 2\\
    Y& 0& 0& 2& 1& 1\\
  \end{tabular}
  \caption{English-Greek context vectors. This can be considered a
    ``view'' of the data. Lets call it biligual view.}
  \label{tab:example1-2}
\end{table}

\end{frame}

\begin{frame}
  \frametitle{Bilingual Data}
  Note that
  \begin{itemize}
    \item 
    \item 
  \end{itemize}
\end{frame}

%------------------------------------------------
\begin{frame}
\frametitle{Monolingual Data}
\begin{table}[htbp]
  \centering
  \begin{tabular}{c c c c}
    A& B \\
    A& W& X \\
    A& X& Y   \\
    W& B& C& Y\\
  \end{tabular}
  \caption{Monolingual English Corpus}
  \label{tab:monolingual1}
\end{table}

\begin{table}[htbp]
  \centering
  \begin{tabular}{c | c c c c c c}
    & A& B& C& W& X& Y\\
    \hline
    A& 3& 1& 0& 1& 1& 0\\
    B& -& 2& 1& 1& 0& 1\\
    C& -& -& 1& 1& 0& 1\\
    W& -& -& -& 2& 1& 1\\
    X& -& -& -& -& 2& 1\\
    Y& -& -& -& -& -& 2\\
  \end{tabular}
  \caption{English-Greek context vectors. This can be considered a
    ``view'' of the data. Lets call it biligual view.}
  \label{tab:example1-2}
\end{table}
\end{frame}
\begin{frame}
\begin{table}[htbp]
  \centering
  \begin{tabular}{c | c c c c c c}
    & A& B& C& W& X& Y\\
    \hline
    A& 6& 5& 5& 1& 1& 2\\
    B& 5& 6& 5& 1& 1& 2\\
    C& 5& 5& 6& 2& 2& 4\\
    W& 1& 1& 2& 6& 5& 5\\
    X& 1& 1& 2& 5& 6& 5\\
    Y& 2& 2& 4& 5& 5& 6\\
  \end{tabular}
  \caption{English-Greek context vectors. This can be considered a
    ``view'' of the data. Lets call it biligual view.}
  \label{tab:example1-2}
\end{table}


\end{frame}

%------------------------------------------------

\begin{frame}
\frametitle{Combining The Two Matrices}
\begin{block}{PPDB}
  Filter B $\times$ B' by M \\
  B * B' provides recall and M provides precision.
\end{block}

\begin{block}{Multiview approach : using CCA}
  Multiview encompasses large number of techniques.\\
  CCA is one of them.
\end{block}
\end{frame}


\end{document} 
