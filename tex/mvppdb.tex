\documentclass[11pt]{article}
\usepackage[ruled,]{algorithm2e}
\usepackage{naaclhlt2015}
\usepackage{times}
\usepackage{mathtools}
\usepackage{url}
\usepackage{latexsym}
\usepackage{amsfonts}
\usepackage{enumitem}
\usepackage{amsmath}
\usepackage{graphicx}
\usepackage{tabularx}
\usepackage{bm}
\usepackage{changepage}
%\usepackage{booktabs}
\usepackage{placeins}
\usepackage[normalem]{ulem}
\usepackage[table]{xcolor}
\usepackage{color, colortbl}
\newcommand{\cwindow}{1, 2, 4, 6, 8, 10, 12, 14, 15}
\newcommand{\cwinlen}{9}
\newcommand{\ctotalview}{16}
\newcommand{\xline}[0]{\noindent\underline{\makebox[0.1cm][l]{}}}
\newcommand{\specialcell}[2][c]{\begin{tabular}[#1]{@{}c@{}}#2\end{tabular}}
\newcommand{\mb}[1]{\textbf{#1}}
\newcommand{\ma}[1]{#1^\dagger}
\newcommand{\mi}[1]{\textbf{#1}}
\newcommand{\y}[1]{#1^*}

%% Pretty fragile code to enable boldness and inheritance of separator.
\let\oldmc\multicolumn
\makeatletter
\newcolumntype{B}[3]{>{\boldmath\DC@{#1}{#2}{#3}}c<{\DC@end}}
\newcommand{\mcinherit}{\renewcommand{\multicolumn}[3]{\oldmc{##1}{##2}{\ifodd\rownum \@oddrowcolor\else\@evenrowcolor\fi ##3}}}
\makeatother

\newcommand{\m}[1]{\multicolumn{1}{c}{#1}}
\newcommand{\mm}[1]{\multicolumn{1}{c|}{#1}}
\newcommand{\my}[1]{\multicolumn{1}{B{.}{.}{-1}}{#1^*}}
\newcommand{\myy}[1]{\multicolumn{1}{B{.}{.}{-1}|}{#1^*}}
\usepackage{array}
\usepackage{enumitem}
\usepackage{dcolumn}
\newcolumntype{H}{>{\setbox0=\hbox\bgroup}c<{\egroup}@{}}
\newcolumntype{d}[1]{D{.}{.}{#1}}
\newcolumntype{S}{l}
\makeatletter
\newcommand{\remove}[1]{}
\newcommand{\removet}[1]{#1}
\newcommand*{\@rowstyle}{}
\newcommand*{\rowstyle}[1]{% sets the style of the next row
  \gdef\@rowstyle{#1}%
  \@rowstyle\ignorespaces%
}

\newcolumntype{=}{% resets the row style
  >{\gdef\@rowstyle{}}%
}

\newcolumntype{+}{% adds the current row style to the next column
  >{\@rowstyle}%
}

\newcommand{\raman}[1]{ (\textcolor{red}{Raman: #1})}

\definecolor{lightgray}{gray}{0.96}
\definecolor{darkgray}{gray}{0.7}
\definecolor{darkergray}{gray}{0.0}
%\setlength\titlebox{5cm}

% You can expand the titlebox if you need extra space
% to show all the authors. Please do not make the titlebox
% smaller than 5cm (the original size); we will check this
% in the camera-ready version and ask you to change it back.

\title{Multiview LSA: Representation Learning via Generalized CCA}

% \author{Pushpendre Rastogi \\
%   Johns Hopkins University \\
%   {\tt pushpendre@jhu.edu} 
% }

\date{} 

\begin{document}
\maketitle
\begin{abstract}
  \emph{Multiview LSA (MVLSA)} is a generalization of Latent Semantic
  Analysis (LSA) that supports the  
  fusion of arbitrary views of data and relies on Generalized Canonical Correlation
  Analysis (GCCA). We present an algorithm
  for fast approximate computation of GCCA, which when coupled with methods
  for handling missing values, is general enough to approximate
  some recent algorithms for inducing vector representations of
  words. Experiments across a comprehensive 
  collection of test-sets show our approach to be competitive with the
  state of the art.   
\end{abstract}

\section{Introduction}
\newcite{winograd1972understanding} wrote that: \emph{``Two sentences
  are paraphrases if they produce the same representation in the
  internal formalism for meaning''}.  This intuition is made soft in
vector-space models \cite{turney2010frequency}, where we say that
expressions in language are paraphrases if
their representations are \emph{close} under some distance measure.

% Raman: Such a distance can be learned directly from natural language text data by first embedding words in a vector space. This is precisely what Latent Semantic Analysis (LSA), one of the earliest linguistic vector space models attempts to achieve. 

One of the earliest linguistic vector space model is Latent
Semantic Analysis (LSA).  LSA has been used widely in fields such
as Information Retrieval and Cognitive Science, but is limited in its
reliance on a single matrix, or \emph{view}, of term co-occurrences.

%% Many recent efforts since the introduction of LSA are similarly
%% restricted to a single view of terms, focusing on some combination of
%% finding an ``optimal'' view, paired with more advanced algorithms\raman{Need references}.


Here we address the single-view limitation of LSA by demonstrating
that the framework of Generalized Canonical Correlation Analysis
(GCCA) can be used to perform Multiview LSA (MVLSA). This
approach allows for the use of an arbitrary number of views in the
induction process, including embeddings induced using other algorithms. We also present a fast approximate method for performing GCCA and approximately recover the objective of \cite{pennington2014glove} while accounting for missing values.

Our experiments show that MVLSA is competitive with recent approaches.
As a methodological aside, we discuss the (in-)appropriateness of
various datasets being used as the basis for comparison within the community. 
% \raman{Perhaps you want to comment on the inappropriateness of conclusions people make on such small datasets. I am sure people do realize that some of the datasets are too small.}

% \section{Motivating LSA, CCA and GCCA}
\section{Motivation} 

% One of the most important challenges in machine learning is to automate the learning of useful representations from raw data. Unsupervised approaches to representation learning aim to capitalize on unlabeled data which is cheap and abundant compared to labeled data which is expensive, time-consuming and hard to obtain. The success of such data-driven approaches depends not only on how much data we can process but also on how well the representations that we learn correlate with the underlying semantic content in the data. To that end, in this paper, we consider multi-view representation learning. 

% One of the most popular and almost ubiquitous representation learning technique is principal component analysis (PCA). In natural language processing, PCA on term-document frequency matrix leads to what is known as latent semantic analysis (LSA). 

LSA is an unsupervised representation learning technique that consists of performing principal component analysis (PCA) on term-document frequency matrix. % Subsequently, these principal directions form the basis for the vector space which captures maximal variation (amongst all vector spaces of the same size) in the input terms-document statistics~\cite{landauer1997solution}. 
The principal directions found by LSA form the basis of the vector-space in which to represent the input terms~\cite{landauer1997solution}. LSA can thus be seen as forming a compact word representation that captures as much {\bf{\emph{{variation}}}} in the input term-document statistics as possible.
% , or equivalently one that is \emph{optimal} for reconstructing the term statistics. 
% Since LSA is simply an application of PCA therefore all the theoretical results for PCA apply to LSA as well.
% In other words, LSA learns the subspace that captures maximal variation 
% Since LSA is simply an application of PCA therefore all the theoretical results for PCA apply to LSA as well.

% Recently, a richer family of representation learning techniques have emerged that leverage multiple views in the data. 
While LSA is an effective and computationally efficient representation learning technique, it is inherently limited to a single view in data. Often, we can learn better representations when we have more than one view in data. 
% multiple views in data. 
The different % multiple 
views may be different modalities or sources of information; for instance, text in different languages, or may be derived from given data as natural splits; for instance, the left and right contexts of words. 
% in many settings, we have more than one view in data; consider for instance, text in different languages, or the left and right contexts of words. 
% These multiple views may be different modalities or sources of information; for instance, text in different languages, or may be derived from given data as natural splits; for instance, the left and right contexts of words. 
% ; for instance, text in different languages, or it is natural to consider splits of data as different views; for instance, the left and right contexts of words. 
In such settings, LSA, and related techniques, would still process data as a single monolithic object obtained by stacking the two views. The alternative ``multi-view'' approach to representation learning that has recently received a lot of attention is based on canonical correlation analysis (CCA). These techniques aim at learning representations that are maximally {\bf{\emph{{correlated}}}} across a pair of views. 
% Multi-view learning techniques based on CCA/GCCA may be viewed as representation learning techniques with soft supervision and can help learn features that are more interpretable and better correlated with the latent semantic content in data. 
CCA based techniques have recently been applied by~\cite{dhillon2011multi,dhillon2012two,faruqui2014improving} for
learning embeddings of words. % Note that CCA is a special case of \emph{MAX-VAR GCCA}, see \cite{velden2011on} for proof.

Much prior work in multi-view learning has focussed on a pair of views (audio+video, acoustic+articulation, bi-text). However, multiple views in data are diverse and abundant. Here, we consider an extension of CCA to more than two views. While there is no standard notion of correlation to more than two sets of covariates, it is natural to consider simple functions of all pairwise correlation such as sum-correlation or max-correlation. These extensions are collectively referred to as Generalized CCA (GCCA); we are in particular interested in the MAX-VAR variant of GCCA as it leads to a tractable and scalable algorithm as we show in the next section. 


% While much work in multi-view learning has focussed on a pair of views (audio+video, acoustic+articulation, bi-text), multiple views are diverse and abundant. 

% has received a lot of attention recently in unsupervised representation learning. Under the assumption that either view has sufficient information about the latent semantic content, multi-view learning reduces the complexity of learning problem by restricting hypothesis class in each view to those which tend to ``agree with each other�� \cite{SridharanKakade:08}. The main insight is that this complexity reduction can happen using unlabeled data which is often plenty, and definitely the case in natural language processing. Furthermore, under the multi-view assumption given above, it was shown that canonical correlation analysis (CCA), a subspace learning technique that looks for representations that are maximally correlated across the two views, is ideal for dimensionality reduction as it loses little predictive power~\cite{Sridharan:08}. CCA can be formulated as follows: given

% leverage additional views by restricting possible representations to those that simultaneously provide a good representation for both views. 
% \emph{optimal} for reconstructing input statistics in both views. 
% Furthermore, under the assumption that either view has sufficient information about the latent semantic content, canonical correlation analysis (CCA), a subspace learning technique that looks for representations that are maximally correlated across the two views, yields such a shared representation~\cite{Sridharan:08}. 

% multi-view learning reduces the complexity of learning problem by restricting hypothesis class in each view to those which tend to ``agree with each other�� \cite{SridharanKakade:08}. The main insight is that this complexity reduction can happen using unlabeled data which is often plenty, and definitely the case in natural language processing. CCA can be formulated as follows: given


% CCA is another dimensionality reduction method that finds two vector-space basis such that once we project two datasets X and Y to those basis then the correlation between corresponding projections gets maximized\cite{hotelling1935the}. CCA has recently been applied by \cite{dhillon2011multi,dhillon2012two,faruqui2014improving} for learning embeddings of words. Note that CCA is a special case of \emph{MAX-VAR GCCA}, see \cite{velden2011on} for proof.

%\section{Motivation}
%\label{sec:motivation}
%GCCA has been derived in many ways and we would use those derivations
%to motivate our approach. \cite{horst1961generalized} derived GCCA as follows. Assume that you have samples of $J$
%``co-variates''. Horst's goal was to find unit variance linear
%projections $\textrm{Z}$ such that some measure of the inter-projection correlation
%matrix $\Phi$ is maximized. For example one could choose the measure to be
%the spectral-norm of $\Phi$. Note that maximizing the spectral-norm of $\Phi$ is equivalent to
%finding $Z$ that can be best explained by a rank 
%one approximation. In other words we are finding $Z$ that are most
%amenable to rank-one PCA, or that can be best explained by a single
%term factor model. \cite{kettenring1971canonical} called this procedure
%\emph{MAX-VAR GCCA} and showed that this
%objective was equivalent to another one proposed by
%\cite{carroll1968generalization}. Carroll's objective was to maximize the following: $\sum_{j=1}^J
%\textrm{correlation}(G, Z_j)^2$. In
%words this expression tells us to find an orthogonal  
%representation $G$ of the co-variates $Z_j$ that is maximally
%correlated to them. This objective agrees with the intuition that representations
%learnt from multiple views should correlate with all of them as much
%as possible.
%
%\cite{bach2005probabilistic} presented a probabilistic
%interpretation for CCA. Though they did not generalize it to
%include GCCA we believe that one could give a probabilistic
%interpretation of \emph{MAX-VAR GCCA} easily and we are working on
%it. We mention it, since a probabilistic
%interpretation would allow us to build an online-generative model and learn
%lexical representations unlike methods like Glove or LSA that rely
%solely on global term cooccurrence matrices and cannot calculate
%perplexity or generate sequences. We also note that
%\cite{via2007learning} presented a neural network model of GCCA and 
%adaptive/incremental GCCA. However, that approach is out of the scope of this work.
%
%Often in many applications we have multiple views in data, for
%instance the left and right contexts of words are two views which are both
%present at training time. In such
%settings, LSA, and related techniques, process data as a single
%monolithic object obtained by stacking different views. However, the
%alternative multi-view approaches like CCA have received attention
%recently in semi-supervised learning because multi-view learning reduces the complexity of learning problem by
%restricting hypothesis class in each view to those which tend to
%agree with each other under the assumption that either
%view has sufficient information about the latent semantic content
% \cite{sridharan2008information}. 
%%% Another benefit of using GCCA over vanilla LSA is that since we can
%%% naturally fuse the statistics generated by using different window sizes we
%%% don't have to specify an arbitrary weighting method, like reciprocal
%%% weighting, for creating a single cooccurrence matrix to represent a
%%% corpus. NOT TRUE, NOT TRUE

\section{Proposed Method: MVLSA}
\label{sec:gcca}
Let $X_j \in \mathbb{R}^{N\times d_j} \;
\forall j \in [1,\ldots,J]$ be the mean centered matrix containing
data from view $j$ such that row $i$ of $X_j$ contains the information for
word $w_i$. Let the number of words in the vocabulary be $N$
and number of contexts (columns in $X_j$) be $d_j$. %% Note that $N$
%% remains the same and $d_j$ varies across views.
Following standard
notation \cite{hastie2009elements} we call $X_j^\top X_j$ the scatter
matrix and $X_j (X_j^\top X_j)^{-1}X_j^\top$ the projection matrix.

The objective of \emph{MAX-VAR GCCA} can be written as the following search problem:
 Find $G \in \mathbb{R}^{N\times r}$ and $U_j \in
\mathbb{R}^{d_j \times r}$ that satisfy expression~\ref{eq:gcca}:
\begin{equation}
  \label{eq:gcca}
\begin{split}
  \operatorname*{\arg\,\min}_{G,U_j} & \sum_{j=1}^J \begin{Vmatrix} G - X_jU_j \end{Vmatrix}^2_F \\
  \text{subject to } & G^\top G = I
\end{split}
\end{equation}
The matrix $G$ that satisfies expression~\ref{eq:gcca} would also be our
vector representation of the vocabulary.
Finding $G$ reduces to spectral decomposition of sum of projection matrices of different views: Define
\begin{align}
P_j =& X_j(X_j^\top X_j)^{-1}X_j^\top \label{eq:pp}\\
M =& \sum_{j=1}^J P_j \label{eq:mm}
\end{align}
Then $G$ and $U_j$ satisfy:
\begin{align}
M G =& G \Lambda\\
U_j =& \left(X_j^\top X_j\right)^{-1} X_j^\top G
\end{align}

%% The above expressions tell us that our word representations are the
%% eigenvectors of the sum of $J$ projection matrices. Also note that the
%% dimensions of $G$ are orthogonal to each other. %% Orthogonality in
%% representations is a nice property that we 
%% will discuss later.

We can immediately see that we can not store
 $P_j \in \mathbb{R}^{N \times N}$ because of memory constraints.
Also the scatter matrices may be non-singular so the procedure may become
ill-posed. 
Let us now describe a method to overcome these problems.

\noindent\textbf{Approximate Regularized GCCA}: GCCA can be regularized by adding $r_jI$ to 
scatter matrix $X_j^\top X_j$ before doing the inversion where
$r_j$ is a small constant e.g. $1e-8$. Equations~\ref{eq:pp}
and \ref{eq:mm} change to
\begin{align}
  \widetilde{P}_{j} =& X_j(X_j^\top X_j+r_jI)^{-1}X_j^\top \label{eq:6}\\
  M =& \sum_{j=1}^J \widetilde{P}_{j} \label{eq:mmm}
\end{align}

We can side step the computational difficulties by assuming that only
the top $m$ singular values of $X_j$ are non-negligible. Parts of the
following method were proposed to us by \cite{savostyanov}.
Let $SVD_m$ denotes a partial SVD where $S_j$ is a rectangular diagonal
matrix that contains only the $m$ largest singular values and $A_j, B_j$
are square, orthonormal, unitary matrices. Defining $SVD_m$ like this
ensures correctness but in practice we only need to compute $m$
columns of $A_j$. Take the SVD of $X_j$.
$$A_{j} S_{j} B^\top_{j} \xleftarrow{SVD_{m}} X_j$$
 Substitute the above in equation~\ref{eq:6} to get 
$$\widetilde{P}_j = A_j S_j^\top(r_j I + S_j S_J^\top)^{-1}S_j A_j^\top$$ 
Define, $T_j \in \mathbb{R}^{m \times m}$ to be the diagonal matrix such that
$T_jT_j^\top = S_j^\top(r_j I + S_j S_J^\top)^{-1}S_j $ then
$$\widetilde{P}_j = A_j T_j T_j^\top A_j^\top$$
Define, $\tilde{M} = \left[ A_1T_1 \ldots A_JT_J \right] \in \mathbb{R}^{N
  \times mJ}$
Then 
$$M = \tilde{M} \tilde{M}^\top$$
Perform QR decomposition of $\tilde{M}$ to get
$$M = Q R R^\top Q$$
Do eigen decomposition of $R R^\top \in \mathbb{R}^{mJ \times mJ}$
to get its eigen vectors $U$ and eigen values $S$.
$$M = Q U S U^\top Q^\top$$
 which implies $G = QU$. 

\subsection{Computing SVD of mean centered $X_j$}
\label{ssec:svdmc}
Recall that we assumed $X_j$ to be mean centered matrices. Let $Z_j
\in \mathbb{R}^{N \times d_j}$ be sparse matrices containing
mean-uncentered cooccurrence counts. Let $f_j = n_j \circ t_j $ be the preprocessing
function that we apply to $Z_j$. 
\begin{align}
  Y_j =& f_j (Z_j) \\
  X_j =& Y_j + 1 (1^\top Y_j)
\end{align}
In order to compute the SVD of mean centered matrices $X_j$ we first
compute the partial SVD of uncentered 
matrix $Y_j$ and then update it. See \cite{brand2006fast} for details.
%% We experimented with representations created from the
%% uncentered matrices $Y_j$ and found that they performed as well as 
%% the mean centered versions but we would not mention them further since
%% it is computationally efficient to follow the principled approach. We
%% note, however, that even the method of mean-centering the SVD
%% produces an approximation.

\subsection{Handling missing rows across views}
\label{ssec:missing}
%% Recall that we assumed that rows of $X_j \forall j \in [0,\ldots , J]$ correspond to unique
%% words in the vocabulary and that the rows correspond to each
%% other.
With real data it may happen that a word was not observed in a view at
all. A large number of 
missing rows can corrupt the learnt representations since the rows
in the left singular matrix become zero.
%% The procedure described above
%% can not recover from this and the representation for those words may become a
%% one hot vector. 
To counter this problem we adopted a variant of the ``missing-data
passive'' algorithm from \cite{van2006generalized} who modified the
GCCA objective to counter the problem of missing 
rows.\footnote{A more recent paper, \cite{van2012generalized},
  describes newer iterative and non-iterative(Test-Equating Method)
  approaches for handling missing values. It is possible that using
  one of those methods could improve performance.}
Specifically, the objective now becomes:
\begin{equation}
  \label{eq:gcca2}
\begin{split}
  \operatorname*{arg\,min}_{G,U_j} & \sum_{j=1}^J \begin{Vmatrix} K_j(G - X_jU_j) \end{Vmatrix}^2_F \\
  \text{subject to } & G^\top G = I
\end{split}
\end{equation}
if row $i$ of view $j$ is observed then $[k_j]_{ii} = 1$ otherwise $0$.
Essentially $K_j$ is a diagonal row-selection matrix which ensures
that we optimize our representations only on the observed rows. Note that
$X_j = K_jX_j$ since the rows that $K_j$ removed were already
zero. Let, $K =
\sum_j K_j$ then the optima
of the objective can be computed by modifying Equation~\ref{eq:mmm} as:
\begin{align}
  M =& K^{-\frac{1}{2}}(\sum_{j=1}^J P_j)K^{-\frac{1}{2}}
\end{align}
Again if we regularize and approximate the GCCA solution, we get
$G=QU$ where $Q, R$ come from the QR decomposition of
$K^{-\frac{1}{2}}\tilde{M}$. Also, we mean center the matrices using
only the observed rows.

Also note that other heuristic weighting schemes could be used
here. For example if we modify our objective as follows then we would
approximately recover the objective of \cite{pennington2014glove}:
\begin{equation}
  \label{eq:gcca3}
\begin{split}
  \operatorname*{arg\,min}_{G,U_j} & \sum_{j=1}^J \begin{Vmatrix} W_j K_j(G - X_jU_j) \end{Vmatrix}^2_F \\
  \text{subject to } G^\top G &= I \\
  \text{where: } [W_j]_{ii} &= \frac{w_i}{w_{\max}}^{\frac{3}{4}} \text{ if } w_i <
  w_{\max} \text{ else } 1 \\
  \text{and } w_i &=  \sum_k [X_j]_{ik}
\end{split}
\end{equation}


\section{Data}
\label{sec:data}
\noindent\textbf{Training Data}: We used the English portion of the \textit{Polyglot} Wikipedia dataset
released by \newcite{al2013polyglot} to create 15 \emph{irredundant} views of
cooccurrence statistics where element $[z]_{ij}$ of view $Z_k$
represents that number of times word $w_j$ occurred $k$ words behind
$w_i$.
%% We lowercased all the words and discarded all
%% words which were longer than 5 characters and contained more than 3 non
%% alphabetical symbols. This was done to preserves years and smaller
%% numbers.
We selected the top 500K words by occurrence to 
create our vocabulary for the rest of the paper.

We extracted cooccurrence statistics from a large bitext corpus that was made by combining a
number of parallel bilingual corpora. See \cite{ganitkevitch2013ppdb} for details and
Table~\ref{tab:dataperlang} for a summary. The Berkeley aligner was used for word alignment. Element
$[z]_{ij}$ of the \textit{bitext} matrix represents the number of times English
word $w_i$ was aligned to the foreign word $w_j$.

We also used the dependency relations in the \textit{Annotated
  Gigaword Corpus}~\cite{annotatedGigaword12} to create 21
views\footnote{Following is the list of dependency relations that we
  used: nsubj, amod, advmod, rcmod, dobj, prep\xline{}of,
  prep\xline{}in, prep\xline{}to, prep\xline{}on, prep\xline{}for,
  prep\xline{}with, prep\xline{}from, prep\xline{}at, prep\xline{}by,
  prep\xline{}as, prep\xline{}between, xsubj, agent, conj\xline{}and,
  conj\xline{}but, pobj. We selected these dependency relations since
  they seemed to be the particularly interesting which could capture
  different aspects of similarity.}  where element $[z]_{ij}$ of view
$Z_d$ represents the number of times word $w_j$ occurred
as the governor of word $w_i$ under dependency relation $d$.

We combined the knowledge of paraphrases present in FrameNet and PPDB by
using the dataset created by \newcite{rastogi2014augmenting} to create a
\textit{FrameNet} view. Element $[z]_{ij}$ of the \textit{FrameNet}
view represents whether word $w_i$ was present in frame
$f_j$. Similarly we combined the knowledge of morphology present in
the \textit{CatVar} database released by \newcite{habash2003catvar} and
\textit{morpha, morphg} released by \newcite{minnen2001applied}.
The morphological views and the frame semantic views were especially
sparse with densities of 0.0003\% and 0.03\%. While we could have utilized more sources of semantic data like
Cooccurrence in WordNet Synsets or Narrative Chains etc, we decided
that the above set would be representative enough to inform us of the
merits or demerits of the MVLSA method and stopped collecting more data.

\begin{table}[htbp]
  \centering
  \rowcolors{1}{}{lightgray}
  \begin{tabular}{lrr}
    Language & Sentences & English Tokens \\
    \hline
    Bitext-Arabic   & 8.8M   & 190M  \\
    Bitext-Czech    & 7.3M   & 17M   \\
    Bitext-German   & 1.8M   & 44M   \\
    Bitext-Spanish  & 11.1M  & 241M  \\
    Bitext-French   & 30.9M  & 671M  \\
    Bitext-Chinese  & 10.3M  & 215M  \\
    Monotext-En-Wiki& 75M    & 1700M 
  \end{tabular}  
  \caption{Portion of data used to create GCCA representations. All
    numbers have unit of Million.}
  \label{tab:dataperlang}
\end{table}

\noindent\textbf{Test Data}: We evaluated the representations on the
word similarity datasets listed in Table~\ref{tab:testlist}. The first
10 datasets in Table~\ref{tab:testlist} were annotated with different
rubrics and rated on different scales. But broadly they all
contain human judgements about how similar two words are.
The ``T-SYN'' and ``T-SEM'' datasets contain 4-tuples of
analogous words and the task is to predict the missing word given the
first three. Both of these are  open vocabulary tasks while TOEFL is a closed
vocabulary task. 
\begin{table*}[ht] \label{tab:testlist}
  \centering
  %\setlength{\tabcolsep}{1pt}
  %\begin{adjustwidth}{0cm}{}
  \rowcolors{1}{}{lightgray}
  %\resizebox{\textwidth}{!}{
  \begin{tabular}{lr | ccc  | ccc | l}
    Acronym & Size  &
    $\sigma_{0.01}^{0.5}$ & $\sigma_{0.01}^{0.7}$ & $\sigma_{0.01}^{0.9}$ &
    $\sigma_{0.05}^{0.5}$ & $\sigma_{0.05}^{0.7}$ & $\sigma_{0.05}^{0.9}$ &
    Reference  \\ 
    \hline

    MEN    & 3000  & 4.2  & 3.2  & 1.8  & 3.0  & 2.3  & 1.3  & \cite{bruni2012distributional}  \\
    RW     & 2034  & 5.1  & 3.9  & 2.3  & 3.6  & 2.8  & 1.6  & \cite{Luong2013morpho}          \\
    SCWS   & 2003  & 5.1  & 4.0  & 2.3  & 3.6  & 2.8  & 1.6  & \cite{Huang2012Improving}       \\
    SIMLEX & 999   & 7.3  & 5.7  & 3.2  & 5.2  & 4.0  & 2.3  & \cite{hill2014simlex}           \\
    WS     & 353   & 12.3 & 9.5  & 5.5  & 8.7  & 6.7  & 3.9  & \cite{finkelstein2001placing}   \\
    MTURK  & 287   & 13.7 & 10.6 & 6.1  & 9.7  & 7.5  & 4.3  & \cite{Radinsky2011word}         \\
    WS-REL & 252   & 14.6 & 11.3 & 6.5  & 10.3 & 8.0  & 4.6  & \cite{agirre2009study}          \\
    WS-SEM & 203   & 16.2 & 12.6 & 7.3  & 11.5 & 8.9  & 5.1  & -Same-As-Above-                 \\
    RG     & 65    & 28.6 & 22.3 & 12.9 & 20.6 & 16.0 & 9.2  & \cite{Rubenstein1965Contextual} \\
    MC     & 30    & 41.7 & 32.7 & 19.0 & 30.6 & 23.9 & 13.8 & \cite{miller1991contextual}     \\ \hline
    T-SYN  & 10675 & -    & -    & 0.95 & -    & -    & 0.68 & \cite{mikolov2013distributed}   \\
    T-SEM  & 8869  & -    & -    & 1.03 & -    & -    & 0.74 & -Same-As-Above-                 \\
    TOEFL  & 80    & -    & -    & 8.13 & -    & -    & 6.63 & \cite{landauer1997solution}
  \end{tabular}
  %}
  \caption{List of test datasets used. The columns headed $\sigma_{p_0}^r$ contain 
    \emph{MRDS}
    values. The rows for accuracy based test sets contain
    $\sigma_{p_0}$ which does not depend on $r$. See
    \S~\ref{ssec:mdrs} for details.} 
  %\end{adjustwidth}
\end{table*}

\subsection{Significance of comparison} \label{ssec:mdrs}
While surveying the literature
we found that performance on the word similarity datasets is typically
reported in terms of the Spearman correlation between the gold ratings
and the cosine distance between normalized embeddings.
We also noted that researchers do not report
measures of significance of the difference between
the Spearman Correlations even for comparisons on small test sets.
E.g. the comparative difference between ratings produced by competing
algorithms reported in \cite{faruqui2014retrofitting}  
could not be significant for the Word Similarity test set released by
\newcite{finkelstein2001placing},
even if we assume that the correlation between the competing methods was as high as 0.9 and set the
pval threshold to be 0.05. There are other examples such as the
comparisons in \cite{hill2014not}. We would also present a simple method to calculate the
\emph{Minimum Required Difference for Significance (MRDS)} for a
dataset which can be used to filter out insignificant differences. We
also hope that the following discussion would discourage the usage of the smaller
datasets.

\noindent\textbf{Minimum Required Difference for Significance (MRDS)}:
Imagine that there exist two lists of ratings produced by algorithms $A, B$ and a list of gold
ratings $T$. Let $r_{AT}$ and $r_{BT}$ and $r_{AB}$ denote the
Spearman correlations between $A, T$ and $B, T$ and $A, B$ respectively. Also let
$\hat{r}_{AT}, \hat{r}_{BT}, \hat{r}_{AB}$ be their empirical
estimates. and assume that $\hat{r}_{BT} > \hat{r}_{AT}$.

For word similarity datasets the MRDS $=\sigma_{p_0}^r$ satisfies the following proposition:
{\small $$ (r_{AB} < r) \land (|\hat{r}_{BT} - \hat{r}_{AT}|{<}\sigma_{p_0}^r)
  {\implies} \textit{pval} > p_0$$}
Where $\textit{pval}$ is the probability of test statistic under the
null hypothesis that $r_{AT} = r_{BT}$ found using the Steiger's test \cite{steiger1980tests}.
Let $\textit{stest}$
denote Steiger's test predicate which satisfies the following:
$\textit{stest}(\hat{r}_{AT}, \hat{r}_{BT}, r_{AB}, p_0, n)
{\implies} \textit{pval} < p_0$. Then we found
$\sigma_{p_0}^r$ by solving the following:
{\small $$\sigma_{p_0}^r = \min\{\sigma | \forall\, 0 {<} r' {<} 1\, \textit{stest}(r',
\min(r'+\sigma, 1), r, p_0, n) \} $$}
Note that MRDS is a liberal threshold and it only guarantees that
differences in correlations below that threshold can never be
significant. MRDS might still mark differences as significant
when they are not but it is useful in reducing some of the noise in
the evaluations.  The values of $\sigma_{p_0}^r$ are shown in
Table~\ref{tab:testlist}.

For the accuracy based test-sets we found MRDS$=\sigma_{p_0}$ that satisfied
the following:
{\small $$ 0< (\hat{\theta}_{B} - \hat{\theta}_{A})<\sigma_{p_0}
  {\implies} \text{p}(\theta_{B} \le \theta_{A}) > p_0$$}
Specifically, we calculated the posterior probability
$\text{p}(\theta_{B} \le \theta_{A})$ with a flat prior of
$\beta(1,1)$ to solve the following:\footnote{instead of using  McNemar's test
  \cite{mcnemar1947note} since the Bayesian approach is tractable and
  more direct. A calculation with $\beta(0.5, 0.5)$ as the prior
  changed $\sigma_{0.5}$ from 6.63 to 6.38 for the TOEFL dataset but
  did not affect MRDS for the T-SEM and T-SYN datasets.}
{\small $\sigma_{p_0}=\min\{\sigma |\forall\,
  0{<}\theta{<}\min(1{-}\sigma,0.9)\,$
  $\text{p}(\theta_{B}{\le} \theta_{A}| \hat{\theta}_A{=}\theta,
\hat{\theta}_B{=}\theta+\sigma, n) < p_0\}$}
Here $\theta_{A}$ and $\theta_{B}$ are probability of correctness of
algorithm A and B and $\hat{\theta}_{A}$ and $\hat{\theta}_{B}$ are
observed empirical accuracies.

Unfortunately, there are no widely followed train-test splits of the above
datasets and we also evaluated the effects of hyper-parameter tuning
on the entire test set therefore our final comparison could have
favored us due to \emph{soft supervision}. %% on these datasets while
%% hyperparameter tuning
However the consistent performance of our
method across the test sets lends hope that the trends we report would
generalize.


\section{Experiments and Results}
\label{sec:exp}
We wanted to answer the following questions through our experiments:
(1) How do hyper parameters affect performance? (2) What is the
contribution of the multiple sources of data to performance? (3) How
does the performance of MVLSA compare with other methods? For brevity we would only show
tuning runs for the larger datasets in this section and distinguish
the top performing configurations in bold using the small threshold
values in column~$\sigma_{0.05}^{0.09}$ of Table~\ref{tab:testlist}.

\noindent\textbf{Effect of Hyper parameters}: 
$f_j$: We modeled the preprocessing function $f_j$ as the composition
of two functions, i.e. $f_j = n_j \circ t_j$.
  $n_j$ represents nonlinear preprocessing that is usually
  employed with LSA. We experimented by setting $n_j$ to be
  Identity, logarithm of count plus one and the Fourth root of the
  count.
  %% \footnote{We also experimented with other powers of the counts (0.12, 0.5
  %% and 0.75) on a smaller dataset and found that the fourth root
  %% performed the best.}
  $t_j$ represents the truncation of columns and can be interpreted as
  a type of regularization of the raw counts themselves through which
  we prune away the noisy contexts. Decrease in $t_j$
  also reduces the influence of views that have a large number of
  context columns and emphasizes the sparser views. 
  Table~\ref{tab:n} and Table~\ref{tab:t} show the results.
\begin{table}[htbp]
  \centering
  \begin{tabular}{=l| +c +c +c}
    Test Set                            & Log  & Count & Count$^{\frac{1}{4}}$ \\ \hline
    MEN                                 & 67.5 & 59.7  & \mb{70.7}                  \\
    RW                                  & 31.1 & 25.3  & \mb{37.8}                  \\
    SCWS                                & 64.2 & 58.2  & \mb{66.6}                  \\\remove{
    SIMLEX                              & 36.7 & 27.0  & \mb{38.0}                  \\
\rowstyle{\color{darkergray}}    WS     & 68.0 & 60.4  & \mb{70.5}                  \\
\rowstyle{\color{darkergray}}    MTURK  & 57.3 & 55.2  & \mb{60.8}                  \\
\rowstyle{\color{darkergray}}    WS-REL & 60.4 & 52.7  & \mb{62.9}                  \\
\rowstyle{\color{darkergray}}    WS-SEM & 75.0 & 67.2  & \mb{76.2}                  \\
\rowstyle{\color{darkergray}}    RG     & 69.1 & 55.3  & \mb{75.9}                  \\
\rowstyle{\color{darkergray}}    MC     & 70.5 & 67.6  & \mb{80.9}                  \\}
    T-SYN                               & 45.7 & 21.1  & \mb{53.6}                  \\
    T-SEM                               & 25.4 & 15.9  & \mb{38.7}                  \\\remove{
  \rowstyle{\color{darkergray}}  TOEFL  & 81.2 & 70.0  & \mb{81.2} }
  \end{tabular}
  \caption{Performance versus $n_j$, the non linear processing of
    Co-occurrence counts.$\, t =200K, \; m=500, \; v=16, \; k=300$.}
  \label{tab:n}
\end{table}

\begin{table}[htbp]
  \centering
  \resizebox{0.5\textwidth}{!}{
  \begin{tabular}{=l | +c +c +c +c H +c H +c}
Test Set                            & 6.25K & 12.5K & 25K  & 50K  & 75K  & 100K & 150K & 200K \\ \hline
MEN                                 & 70.2  & \mi{71.2}  & \mi{71.5} & \mi{71.6} & \mi{71.4} & \mi{71.2} & \mi{71.0} & \mi{70.7} \\   
RW                                  & \mi{41.8}  & \mi{41.7}  & \mi{41.5} & \mi{40.9} & \mi{40.7} & 39.6 & 38.3 & 37.8 \\ 
SCWS                                & \mi{67.1}  & \mi{67.3}  & \mi{67.1} & \mi{67.0} & \mi{67.3} & \mi{66.9} & \mi{66.8} & \mi{66.6} \\ \remove{
SIMLEX                              & 42.7  & \mb{42.4}  & 41.9 & 41.3 & 40.5 & 39.5 & 38.4 & 38.0 \\ 
\rowstyle{\color{darkergray}}WS     & 68.1  & 70.8  & 71.6 & 71.2 & 71.3 & 70.2 & 70.8 & 70.5 \\ 
\rowstyle{\color{darkergray}}MTURK  & 62.5  & 59.7  & 59.2 & 58.6 & 58.3 & 60.3 & 61.0 & 60.8 \\ 
\rowstyle{\color{darkergray}}WS-REL & 60.8  & 65.1  & 65.7 & 64.8 & 65.2 & 63.7 & 63.7 & 62.9 \\ 
\rowstyle{\color{darkergray}}WS-SEM & 77.8  & 78.8  & 78.8 & 78.2 & 77.7 & 76.5 & 77.0 & 76.2 \\ 
\rowstyle{\color{darkergray}}RG     & 72.7  & 74.4  & 74.7 & 75.0 & 75.0 & 74.3 & 75.6 & 75.9 \\ 
\rowstyle{\color{darkergray}}MC     & 75.2  & 75.9  & 79.9 & 80.3 & 81.0 & 76.9 & 79.6 & 80.9 \\}
T-SYN                               & 59.2  & \mi{60.0}  & \mi{59.5} & 58.4 & 57.4 & 56.1 & 54.3 & 53.6 \\
T-SEM                               & 37.7  & \mi{38.6}  & \mi{39.4} & \mi{39.2} & \mi{39.4} & 38.4 & \mi{38.8} & \mi{38.7} \\\remove{
\rowstyle{\color{darkergray}}TOEFL  & 88.8  & 87.5  & 85.0 & 83.8 & 83.8 & 83.8 & 82.5 & 81.2}
      \end{tabular}
  }
  \caption{Performance versus the truncation threshold, $t$, of raw
    cooccurrence counts. We used $n_j=\textrm{Count}^{\frac{1}{4}}$
    and other settings were the same as Table~\ref{tab:n}.} 
  \label{tab:t}
\end{table}
$m$: The number of left singular vectors extracted after SVD of the preprocessed cooccurrence
  matrices can again be interpreted as a type of regularization, since
  the result of this truncation is that we find cooccurrence patterns 
  only between the top left singular vectors. We set $m_j = max(d_j,
  m)$ with $m=[100, 300, 500]$. See table~\ref{tab:n}.

\begin{table}[htbp]
  \centering
  \begin{tabular}{=l | +c +c +c +c}
Test Set                            & 100  & 200  & 300  & 500  \\\hline
MEN                                 & 65.6 & 68.5 & \mi{70.1} & \mi{71.1} \\
RW                                  & 34.6 & \mi{36.0} & \mi{37.2} & \mi{37.1} \\
SCWS                                & 64.2 & \mi{65.4} & \mi{66.4} & \mi{66.5} \\\remove{
SIMLEX                              & 38.4 & 40.6 & \mb{41.1} & 40.3 \\
\rowstyle{\color{darkergray}}WS     & 60.4 & 67.1 & 69.4 & \mb{71.1} \\
\rowstyle{\color{darkergray}}MTURK  & 51.3 & 58.3 & 58.4 & \mb{58.9} \\
\rowstyle{\color{darkergray}}WS-REL & 49.0 & 58.2 & 61.6 & \mb{65.1} \\
\rowstyle{\color{darkergray}}WS-SEM & 73.6 & 76.8 & 76.8 & \mb{78.0} \\
\rowstyle{\color{darkergray}}RG     & 61.6 & 69.7 & 73.2 & \mb{74.6} \\
\rowstyle{\color{darkergray}}MC     & 65.6 & 74.1 & \mb{78.3} & 77.7 \\}
T-SYN                               & 50.5 & \mi{56.2} & \mi{56.4} & \mb{56.4} \\
T-SEM                               & 24.3 & 31.4 & 34.3 & \mb{40.6} \\\remove{
\rowstyle{\color{darkergray}} TOEFL & 80.0 & 81.2 & \mb{82.5} & 80.0}
  \end{tabular}                                        
  \caption{Performance versus $m$, the number of left     
singular vectors extracted from raw cooccurrence counts. We set
$n_j=\textrm{Count}^\frac{1}{4}, \; t=100K, \; v=25, \;
k=300$.} 
  \label{tab:m}
\end{table}

$k$: Table~\ref{tab:k} demonstrates the variation in performance
versus the dimensionality of the learnt vector representations of the
  words. Since the dimensions of the MVLSA representations are
  orthogonal to each other therefore creating lower dimensional
  representations is a trivial matrix slicing operation and does not
  require retraining.
  \begin{table}[htbp]
    \centering
  \begin{tabular}{=l | +c H +c +c +c +c +c}
Test Set                            & 10   & 25   & 50   & 100  & 200       & 300       & 500       \\\hline
MEN                                 & 49.0 & 59.3 & 67.0 & \mb{69.7} & \mb{70.2} & \mi{70.1} & \mb{69.8}\\
RW                                  & 28.8 & 33.1 & 33.3 & 35.0 & 35.2      & \mb{37.2} & \mi{38.3} \\
SCWS                                & 57.8 & 62.8 & 64.4 & \mi{65.2} & \mi{66.1}      & \mb{66.4} & \mi{65.1}      \\\remove{
SIMLEX                              & 24.0 & 30.1 & 33.9 & 36.1 & 38.9      & 41.1      & \mb{42.0} \\
\rowstyle{\color{darkergray}}WS     & 46.8 & 57.5 & 63.4 & 69.5 & 69.5      & 69.4      & 66.0      \\
\rowstyle{\color{darkergray}}MTURK  & 54.6 & 65.9 & 67.7 & 61.6 & 60.5      & 58.4      & 57.4      \\
\rowstyle{\color{darkergray}}WS-REL & 38.4 & 49.5 & 55.8 & 63.1 & 62.4      & 61.6      & 56.3      \\
\rowstyle{\color{darkergray}}WS-SEM & 55.3 & 64.7 & 69.9 & 76.9 & 77.1      & 76.8      & 75.6      \\
\rowstyle{\color{darkergray}}RG     & 48.8 & 60.5 & 66.1 & 69.7 & 75.1      & 73.2      & 72.5      \\
\rowstyle{\color{darkergray}}MC     & 37.0 & 57.5 & 59.0 & 71.3 & 79.1      & 78.3      & 75.7      \\}
T-SYN                               & 9.0  & 28.4 & 41.2 & 52.2 & 55.4      & \mb{56.4} & 54.4      \\
T-SEM                               & 2.5  & 10.8 & 21.8 & 34.8 & \mb{35.8} & 34.3      & 33.8      \\\remove{
\rowstyle{\color{darkergray}} TOEFL & 57.5 & 73.8 & 72.5 & 76.2 & 81.2      & 82.5      & 85.0}
  \end{tabular}
  \caption{Performance versus $k$, the final dimensionality of the
    embeddings. We set $ m=300$ and other settings were same as Table~\ref{tab:m}.}
  
  \label{tab:k}
\end{table}

$v$: Recall that in Expression~\ref{eq:gcca3} we described a method to
  set $W_j$. We experimented with a different, more global, heuristic to
  set $[W_j]_{ii} = (K_{ww} \ge v)$. Essentially we removed all
  words that did not appear in $v$ views before doing
  GCCA. Table~\ref{tab:v} shows that changes in $v$ are largely
  inconsequential for performance. In absence of clear evidence in favor of regularization we
  decided to regularize as little as possible and chose $v=16$.   
  \begin{table}[htbp]
    \centering
  \begin{tabular}{=l | +c +c H +c H +c H +c}
Test Set                            & 16   & 17   & 19   & 21   & 23   & 25   & 27   & 29   \\ \hline
MEN                                 & \mb{70.4} & \mb{70.4} & \mi{70.2} & \mi{70.2} & \mi{70.1} & \mi{70.1} & \mi{70.0} & \mi{70.0} \\
RW                                  & \mb{39.9} & \mi{38.8} & \mi{40.1} & \mi{39.7} & 38.3 & 37.2 & 35.3 & 33.5 \\
SCWS                                & \mb{67.0} & \mb{66.8} & \mb{66.8} & \mb{66.5} & \mb{66.3} & \mb{66.4} & \mb{66.1} & \mb{65.7} \\\remove{
SIMLEX                              & 40.7 & 41.0 & 41.1 & \mb{41.2} & 41.2 & 41.1 & 41.1 & 41.0 \\
\rowstyle{\color{darkergray}}WS     & 69.5 & 69.4 & 69.5 & 69.5 & 69.4 & 69.4 & 69.3 & 69.1 \\
\rowstyle{\color{darkergray}}MTURK  & 59.4 & 59.2 & 59.3 & 59.2 & 58.7 & 58.4 & 58.0 & 58.0 \\
\rowstyle{\color{darkergray}}WS-REL & 62.1 & 61.9 & 62.1 & 62.3 & 61.9 & 61.6 & 61.4 & 61.1 \\
\rowstyle{\color{darkergray}}WS-SEM & 76.8 & 76.8 & 76.9 & 77.0 & 76.7 & 76.8 & 76.7 & 76.8 \\
\rowstyle{\color{darkergray}}RG     & 73.0 & 72.8 & 72.7 & 72.8 & 73.6 & 73.2 & 73.4 & 73.7 \\
\rowstyle{\color{darkergray}}MC     & 75.0 & 76.0 & 76.4 & 76.5 & 78.2 & 78.3 & 78.6 & 78.6 \\}
T-SYN                               & \mb{56.0} & \mb{55.8} & \mb{56.0} & \mb{55.9} & \mb{56.3} & \mb{56.4} & \mb{56.3} & \mb{56.0} \\
T-SEM                               & \mb{34.6} & \mb{34.3} & \mb{34.1} & \mb{34.0} & \mb{34.5} & \mb{34.3} & \mb{34.4} & \mb{34.3} \\\remove{
\rowstyle{\color{darkergray}} TOEFL & 85.0 & 85.0 & 85.0 & 83.8 & 83.8 & 82.5 & 82.5 & 80.0}
    \end{tabular}
  \caption{Performance versus minimum view support threshold $v$, The other
      hyperparameters were $n_j=\textrm{Count}^{\frac{1}{4}}, \;
      m=300, \; t=100K$. Though a clear best setting did not emerge,
      we chose $v=25$ as the middle ground.}
  \label{tab:v}
\end{table}
  
$r_j$: The regularization parameter ensures that all the
  inverses exist at all points in our method. We found that the
  performance of our  procedure was invariant to $r$ over a large
  range from 1 to 1e-10. This was because even the 1000th singular
  value of  our data was much higher than 1.%%  which is
  %% consistent with the observation that cooccurrence datasets in NLP
  %% tend to have gently sloping spectrum. For all the experiments in
  %% this paper we set $r=1e-5$.


\begin{table*}[ht]
  \centering
  \label{tab:j}
   \setlength\tabcolsep{3pt}
  \begin{tabular}{=l| +c +c +c +c +c +c +c +c}
Test Set              & \specialcell{All\\Views} & !Framenet &
!Morphology & !Bitext & !Wikipedia & !Dependency &
\specialcell{!Morphology\\!Framenet} &
\specialcell{!Morphology\\!Framenet\\!Bitext} \\\hline
MEN                                 & \mb{70.1} & \mi{69.8} & \mi{70.1} & \mi{69.9} & 46.4 & 68.4 & \mi{69.5} & 68.4 \\
RW                                  & \mb{37.2} & \mi{36.4} & \mi{36.1} & 32.2 & 11.6 & 34.9 & 34.1 & 27.1 \\
SCWS                                & \mb{66.4} & \mi{65.8} & \mi{66.3} & 64.2 & 54.5 & \mi{65.5} & \mi{65.2} & 60.8 \\\remove{
SIMLEX                              & 41.1 & 40.1 & 41.1 & 37.8 & 32.4 & \mb{44.1} & 38.9 & 34.4 \\
\rowstyle{\color{darkergray}}WS     & 69.4 & 69.1 & 69.2 & 67.6 & 43.1 & 70.5 & 69.3 & 66.6 \\
\rowstyle{\color{darkergray}}MTURK  & 58.4 & 58.3 & 58.6 & 55.9 & 52.7 & 59.8 & 57.9 & 55.3 \\
\rowstyle{\color{darkergray}}WS-REL & 61.6 & 61.5 & 61.4 & 59.4 & 38.2 & 63.5 & 62.5 & 58.8 \\
\rowstyle{\color{darkergray}}WS-SEM & 76.8 & 76.3 & 76.7 & 75.9 & 48.1 & 75.7 & 75.8 & 73.1 \\
\rowstyle{\color{darkergray}}RG     & 73.2 & 72.0 & 73.2 & 73.7 & 45.0 & 70.8 & 71.9 & 74.0 \\
\rowstyle{\color{darkergray}}MC     & 78.3 & 75.7 & 78.2 & 78.2 & 46.5 & 77.5 & 76.0 & 80.2 \\}
T-SYN                               & \mb{56.4} & \mi{56.3} & \mi{56.2} & 51.2 & 37.6 & 50.5 & 54.4 & 46.0 \\
T-SEM                               & 34.3 & 34.3 & 34.3 & \mb{36.2} & 4.1  & 35.3 & 34.5 & 30.6 \\\remove{
\rowstyle{\color{darkergray}}TOEFL  & 82.5 & 82.5 & 82.5 & 71.2 & 45.0 & 85.0 & 82.5 & 65.0   }
  \end{tabular}
  \parbox{\textwidth}{\caption{Performance versus views removed from
      the multiview GCCA procedure. !Framenet means that the view
      containing counts derived from Frame semantic dataset was
      removed. Other columns are named similarly. The other
      hyperparameters were $n_j=\textrm{Count}^{\frac{1}{4}}, \;
      m=300, \; t=100K, \; v=25, \; k=300$. }}
\end{table*}
  
\noindent\textbf{Contribution of different sources of data}:
 Table~\ref{tab:j} shows an ablative analysis of performance where we
 remove individual views or some combination of them and measure the
 performance.  It is clear by comparing the last column to the second
 column that adding in more views 
 improves performance. Also we can see that the Dependency based views and the Bitext
 based views give a larger boost than the morphology and FrameNet
 based views, probably because the latter are so sparse.

\noindent\textbf{Comparison to other word representation creation methods:}
There are a large number of methods of creating representations both
multilingual and monolingual. There are many new methods such as
\newcite{yu2014improving,faruqui2014retrofitting,felix2014learning,weston2014hash}
that are performing multiview learning and could be considered as
baselines however it is not straight forward to use those systems to
handle the variety of data that we are using. Therefore, we directly
compare our method to the Glove and the SkipGram model of Word2Vec as
the performance of those systems is considered state of the art.  
We trained the two systems on the English portion of the
\textit{Polyglot} Wikipedia dataset.\footnote{More specifically
we explicitly provided the vocabulary file to Glove and Word2Vec and set the
truncation threshold for word2Vec to 10. Also Glove was trained for 25
iterations. Glove was provided a window of 15 previous words and Word2Vec
used a symmetric window of 10 words.} We also combined their outputs
using MVLSA again to create \emph{G-WSG Combo} embeddings.

We trained our best MVLSA system with data from all the views and by
using the individual best settings of 
the hyper-parameters. Specifically the configuration we used was as
follows: $n_j = \text{Count}^\frac{1}{4}, t=12.5K, m=500, k=300,
v=16$. To make a fair comparison we also provide 
results where we used only the views derived from the \textit{Polyglot}
Wikipedia corpus. See column \emph{MVLSA (All Views)} and \emph{MVLSA
  (Wiki)} respectively. It is clearly visible that MVLSA on the
monolingual data itself is competitive with Glove but
worse than Word2Vec on the word similarity datasets and it is
substantially worse than both the systems on the T-SYN 
and T-SEM datasets. However with the addition of multiple views MVLSA
makes substantial gains, shown in column \emph{MV Gain}, and after consuming the Glove and WSG
embeddings it again improves performance by some margins, as shown in
column \emph{G-WSG Gain}, and outperforms the original systems. 
Using GCCA itself for system combination provides closure
for the MVLSA algorithm since more and more algorithms can now be
combined by using the same method. Finally we contrast the Spearman
correlations $r_s$ with Glove and Word2Vec before and after including
them in the GCCA procedure. The values demonstrate that including Glove and WSG
  during GCCA actually increased the  correlation between them and the
  learnt embeddings, which was agrees with our motivation for
  performing GCCA in the first place.

\begin{table*}[ht]
  \centering
    %\begin{adjustwidth}{0cm}{}
      \rowcolors{1}{}{lightgray}  \mcinherit
  \setlength\tabcolsep{2.2pt}
  \begin{tabular}{=l|                  +d{2.1}     +d{2.1}     +d{2.1}  |   +d{2.1}     +d{2.1}        +d{2.1}     +d{2.1}         +d{2.1}  | +c                                  +c |     +c            +c }
    \mm{Test Set}                  & \m{Glove} & \m{WSG}   & \mm{G-WSG} & \m{MVLSA} & \m{MVLSA}     & \m{MV}    & \m{MV-G-WSG} & \mm{G-WSG} & \multicolumn{2}{c|}{$r_s$ MVLSA} & \multicolumn{2}{c}{$r_s$ MV-G-WSG} \\
                \mm{}              & \m{}      & \m{}      & \mm{Combo} & \m{Wiki } & \m{All Views} & \m{Gain}  & \m{Combined} & \mm{Gain } & \m{Glove}                        & \mm{WSG} & \m{Glove} & \m{WSG}     \\\hline
MEN                                & 70.4      & 73.9      & \myy{76.0} & 71.4      & 71.2          & -0.2      & \y{75.8}     & \ma{4.6}   & 71.9                             & 89.1     & 85.8      & 92.3        \\
RW                                 & 28.1      & 32.9      & 37.2       & 29.0      & \my{41.7}     & \ma{12.7} & \y{40.5}     & -1.2       & 72.3                             & 74.2     & 80.2      & 75.6        \\
SCWS                               & 54.1      & 65.6      & 60.7       & 61.8      & \my{67.3}     & \ma{5.5}  & \y{66.4}     & -0.9       & 87.1                             & 94.5     & 91.3      & 96.3        \\
SIMLEX                             & 33.7      & 36.7      & 41.1       & 34.5      & \y{42.4}      & \ma{7.9}  & \my{43.9}    & 1.5        & 62.4                             & 78.2     & 79.3      & 86.0        \\
WS                                 & 58.6      & \my{70.8} & \y{67.4}   & \y{68.0}  & \my{70.8}     & \ma{2.8}  & \y{70.1}     & -0.7       & 72.3                             & 88.1     & 81.8      & 91.8        \\
MTURK                              & \y{61.7}  & \my{65.1} & 59.8       & 59.1      & 59.7          & 0.6       & \y{62.9}     & 3.2        & 80.0                             & 87.7     & 87.3      & 92.5        \\
WS-REL                             & 53.4      & \my{63.6} & 59.6       & 60.1      & \y{65.1}      & \ma{5.0}  & \y{63.5}     & -1.6       & 58.2                             & 81.0     & 69.6      & 85.3        \\
WS-SEM                             & 69.0      & \y{78.4}  & \y{76.1}   & \y{76.8}  & \y{78.8}      & 2.0       & \my{79.2}    & 0.4        & 74.4                             & 90.6     & 83.9      & 94.0        \\
\rowstyle{\color{darkergray}}RG    & \y{73.8}  & \y{78.2}  & \y{80.4}   & 71.2      & \y{74.4}      & 3.2       & \my{80.8}    & \ma{6.4}   & 80.3                             & 90.6     & 91.8      & 92.9        \\
\rowstyle{\color{darkergray}}MC    & \y{70.5}  & \y{78.5}  & \myy{82.7} & \y{76.6}  & \y{75.9}      & -0.7      & \y{77.7}     & 2.8        & 80.1                             & 94.1     & 91.4      & 95.8        \\
T-SYN                              & 61.8      & 59.8      & 51.0       & 42.7      & 60.0          & \ma{17.3} & \my{64.3}    & \ma{4.3}   &                                  &          &           &             \\
T-SEM                              & \my{80.9} & 73.7      & 73.5       & 36.2      & 38.6          & \ma{2.4}  & 77.2         & \ma{38.6}  &                                  &          &           &             \\
\rowstyle{\color{darkergray}}TOEFL & \y{83.8}  & 81.2      & \y{86.2}   & 78.8      & \y{87.5}      & \ma{8.7}  & \my{88.8}    & 1.3        &                                  &          &           & 
  \end{tabular}
  \caption{Comparison of Multiview LSA against Glove and WSG(Word2Vec
    Skip Gram) algorithms. Using $\sigma_{0.05}^{0.9}$ as the
   threshold we marked the top performing systems with $^*$ and
   highlighted the absolute best in bold. We also marked significant
   increments in performance due to use of additional views with $^\dagger$.
    %%We use $\sigma_{0.05}^{0.9}$ as the
  %% threshold for highlighting the top performing systems in bold and for
  %% marking significant increments due to use of additional views with
  %% asterisk.
  }
  \label{tab:c}
  %\end{adjustwidth}
\end{table*}

\section{Previous Work}
\label{sec:previouswork}
Vector space representations of words have been created using diverse
 frameworks ranging from Spectral methods
 \cite{dhillon2011multi,dhillon2012two}
 %% \footnote{\url{cis.upenn.edu/~ungar/eigenwords}}
 to Neural Networks
 \cite{mikolov2013efficient,mikolov2013distributed,collobert2013word}
 %%\footnote{\url{code.google.com/p/word2vec},\url{metaoptimize.com/projects/wordreprs}}
 and trained using either one
 \cite{pennington2014glove}
 %% \footnote{\url{nlp.stanford.edu/projects/glove}}
 or two sources of cooccurrence statistics
 \cite{zou2013bilingual,faruqui2014improving,bansal2014tailoring,levy2014dependency}
 %% \footnote{\url{ttic.uchicago.edu/~mbansal/data/syntacticEmbeddings.zip,cs.cmu.edu/~mfaruqui/soft.html}}
 or using multi-modal data
 \cite{felix2014learning,bruni2012distributional}.
 
\cite{dhillon2011multi,dhillon2012two} were the first to use
CCA as the primary method to learn vector representations and
\cite{faruqui2014improving} further demonstrated incorporating bilingual
data through CCA improved performance. More recently this same
phenomenon was reported by \newcite{hill2014not} through their
experiments over neural representations learnt from MT systems.
%% Outside of the NLP
%% community \cite{sun2013generalized,tripathi2011data} are two
%% publications 
%% that we are aware of that have used GCCA for ``data fusion''.
And various other researchers have tried to improve the
performance of their paraphrase systems or vector space models by using
diverse sources of information such as bilingual
corpora~\cite{bannard2005paraphrasing,Huang2012Improving,zou2013bilingual}, 
structured datasets~\cite{yu2014improving,faruqui2014retrofitting} or even
tagged images~\cite{bruni2012distributional}; 
%% The intuitive reason that using multiple sources of data improves performance is 
%% that the views complement each other. For example it was mentioned in
%% \cite{ganitkevitch2013ppdb} that monolingual data can't distinguish
%% between antonyms but bilingual data can. And bilingual data confounds
%% words that occur in the same sentence but monolingual data can
%% distinguish them based on their context.
However, the previous
work did not adopt the general, simplifying view that 
all of these sources of data are just cooccurrence 
statistics coming from different sources with underlying latent
factors.\footnote{Though \cite{faruqui2014retrofitting} used
  the sophisticated technique of belief propagation the graph that
  they used it on was an undirected weighted graph that can be
  perfectly represented as an adjacency matrix which is another type of
cooccurrence matrix. Also their framework could not fuse arbitrary views such as
other vector representations.}


GCCA has been derived in many ways and we would use those derivations
to motivate our approach. \cite{horst1961generalized} derived GCCA as follows. Assume that you have samples of $J$
``co-variates''. Horst's goal was to find unit variance linear
projections $\textrm{Z}$ such that some measure of the inter-projection correlation
matrix $\Phi$ is maximized. For example one could choose the measure to be
the spectral-norm of $\Phi$. Note that maximizing the spectral-norm of $\Phi$ is equivalent to
finding $Z$ that can be best explained by a rank 
one approximation. In other words we are finding $Z$ that are most
amenable to rank-one PCA, or that can be best explained by a single
term factor model. \cite{kettenring1971canonical} called this procedure
\emph{MAX-VAR GCCA} and showed that this
objective was equivalent to another one proposed by
\cite{carroll1968generalization}. Carroll's objective was to maximize the following: $\sum_{j=1}^J
\textrm{correlation}(G, Z_j)^2$. In
words this expression tells us to find an orthogonal  
representation $G$ of the co-variates $Z_j$ that is maximally
correlated to them. This objective agrees with the intuition that representations
learnt from multiple views should correlate with all of them as much
as possible.

\cite{bach2005probabilistic} presented a probabilistic
interpretation for CCA. Though they did not generalize it to
include GCCA we believe that one could give a probabilistic
interpretation of \emph{MAX-VAR GCCA} easily and we are working on
it. We mention it, since a probabilistic
interpretation would allow us to build an online-generative model and learn
lexical representations unlike methods like Glove or LSA that rely
solely on global term cooccurrence matrices and cannot calculate
perplexity or generate sequences. We also note that
\cite{via2007learning} presented a neural network model of GCCA and 
adaptive/incremental GCCA. However, that approach is out of the scope of this work.

Often in many applications we have multiple views in data, for
instance the left and right contexts of words are two views which are both
present at training time. In such
settings, LSA, and related techniques, process data as a single
monolithic object obtained by stacking different views. However, the
alternative multi-view approaches like CCA have received attention
recently in semi-supervised learning because multi-view learning reduces the complexity of learning problem by
restricting hypothesis class in each view to those which tend to
agree with each other under the assumption that either
view has sufficient information about the latent semantic content
 \cite{sridharan2008information}. 
%% Another benefit of using GCCA over vanilla LSA is that since we can
%% naturally fuse the statistics generated by using different window sizes we
%% don't have to specify an arbitrary weighting method, like reciprocal
%% weighting, for creating a single cooccurrence matrix to represent a
%% corpus. NOT TRUE, NOT TRUE


\remove{
In the sense of being an application of multiview
learning methods to NLP our work is an addition to the long chain which
started from the work of \cite{yarowsky1995unsupervised} and continued
with \emph{Co-Training}~\cite{blum1998combining}, \emph{CoBoosting}~\cite{collins1999unsupervised} and \emph{2 view
perceptrons}~\cite{brefeld2006efficient}.  CCA is also an algorithm
for multi view learning \cite{kakade2007multi,ganchevuai08}, and it
has a probabilistic interpretation \cite{bach2005probabilistic} as
well.



\section{Future Work}
\label{sec:futurework}
In a rough order of importance we believe we could improve our method in the following ways:
\begin{itemize}[leftmargin=*]
  \itemsep-0.1em
  \renewcommand\labelitemi{--}
  \item We were
  not able to set $v=0$ since our current implementation of GCCA
  procedure requires us to load the entire $\tilde{M}$ matrix in
  memory which is 45*500*500K*8 Bytes = 90GB.
  %% While this is certainly
  %% possible with current hardware a more memory efficient approach is
  %% desirable.
  This is a drawback of our current implementation and we
  are working on implementing a more scalable version of this
  algorithm so that we can run experiments with larger
  vocabularies. For our current experiments the largest vocabulary we
  used consisted of 361K words and had 100\% recall on all the test
  sets except for RW on which the recall was 92\%.
  %% Also note that though the
  %% size of $G$ decreases as we increase $v$ (and hence the recall of
  %% our embeddings) for these experiments the recall of the vocabulary
  %% over the test sets remained complete except for RW and T-SEM for
  %% which the recall was 1700 out of 2034 and 8714 out of 8869
  %% respectively.
\item By implementing the probabilistic version of GCCA which would
  allow us to create generative models which could be trained in an
  online fashion. We conjectured that ``MAXVAR'' formulation of GCCA is closely related to a probabilistic interpretation of GCCA. We are not aware of any work
that has made this connection before and we are working on proving this where the proof technique and style would closely mirror that in \cite{bach2005probabilistic}.
\item By using count dependent non-linear weighting as exemplified
  through Expression~\ref{eq:gcca3}
\item By implementing procedures for constant memory QR decomposition so that we can scale our method to larger vocabularies. 
\item By adding more views such as views that derived from successive
  words instead of  using only the previous words. Since we are not using
  any PMI-like or frequency like features there is scope for
  improvement on that front.  We note that the performance of our
  method on the ``T-SEM'' dataset was very low which we believe could
  have been caused due to lack of PMI type features have been reported
  to work well in \cite{levy2014neural}.  
%% \item by using kernel methods that promise to liberate us from tuning over non-linearities, though tuning over kernels is a problem as well.
\item By using more sophisticated method for handling missing values
  as mentioned earlier.
\item By performing discriminative optimization of multiplicative
  factors over the views. For example we saw that bitext views hurt
  performance on the ``T-SEM'' task whereas they improved performance
  on all the other tasks in general. A simple technique could be to
  simply assign a multiplicative factor to each view and 
  then to tune the values of that factor by using discriminative
  techniques. Of course such a method would not remain unsupervised any
  more. Such a discriminative optimization could help us understand
  better the reasons why a particular corpus improves performance on
  certain datasets but decreases performance on another dataset. %% Since
  %% we could calculate gradient of the gram matrix between words in a
  %% dataset versus the multiplicative factor.
\end{itemize}
}

\section{Conclusion}
While previous work had already demonstrated that incorporating two views
was beneficial this work leveraged that intuition to a
logical extreme and used GCCA to learn distributed 
representations using data from 46 views! Through the results in
Table~\ref{tab:c} we demonstrated that fusing 
multiple data sources through the procedure of MVLSA was effective
in improving the performance of word representations on a
comprehensive set of test datasets. In order to
 perform GCCA over vocabularies with up to 500K words
we presented a new fast algorithm, and also showed that a close
variant of the Glove objective proposed by 
\newcite{pennington2014glove} could be derived as a heuristic for
handling missing data under the MVLSA framework. We also
demonstrated that MVLSA could be used even over a monolingual
dataset thereby providing a more principled alternative of LSA that
removes the need for heuristically combining word-word cooccurrence 
matrices into a single matrix. Finally, while surveying the literature
we noticed that not enough emphasis was being given towards
establishing the significance of 
comparative results and proposed a simple statistic \emph{(MRDS)}
to filter out insignificant comparative gains between competing
algorithms.

\remove{
\section*{Acknowledgments}
This material is based on research sponsored by Defense Advanced Research
Projects Agency (DARPA) under the Deep Exploration and
Filtering of Text (DEFT) Program (Agreement number
FA8750-13-2-0017). We also thank Juri Ganitkevitch for 
providing the word aligned bitext corpus.
}
\bibliographystyle{naaclhlt2015}
\bibliography{references}
\end{document}

